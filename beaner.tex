\documentclass{beamer}
\usepackage[utf8]{inputenc}
\usepackage[]{polski}
\mode<beamer>{
\usetheme{Frankfurt}
\setbeamertemplate{navigation symbols}{}
}
\mode<handout>{
\usepackage{pgfpages}
\pgfpagesuselayout{4 on 1}[a4paper, border shrink=10mm, landscape]
\usetheme{default}
}

\title{Capacitated Vehicular Routing Problem}
\subtitle{Problem rozwożenia kontenerów}
\author{Aleksander Balicki, Łukasz Zapart}
\institute{Instytut Informatyki}
\date{\today}

\begin{document}
\mode*
\begin{frame}
\titlepage
\end{frame}

\begin{frame}
\tableofcontents[hideallsubsections]
\end{frame}


\section{Specyfikacja problemu}

\begin{frame}
\subsection{Oryginalna specyfikacja}
\frametitle{Oryginalna specyfikacja}
	Na placu znajduje sie pewna liczba kontenerów, które należy rozwieść samochodem do odbiorców. Samochód jednocześnie zabiera 3 kontenery. Zaplanować trasy samochodu, tak aby liczba przejechanych kilometrów była minimalna. (Jak rozwiązać ten problem, gdy mamy do dyspozycji kilka samochodów?)
\end{frame}

\begin{frame}
\subsection{Uściślenie specyfikacji}
\frametitle{Uściślenie specyfikacji}
	Na $n$-wierzchołkowym grafie z wagami na krawędziach i wyróżnionym wierzchołkiem $S$, musimy znaleść $k$ prawie rozłącznych cykli (wszystkie spotykają się w wierzchołku $S$), pokrywających cały graf.

	$n$ - liczba odbiorców
	$k$ - liczba ciężarówek

	Wymaga to założenia, że:
	\begin{itemize}
		\item żaden pakunek nie będzie podzielny,
		\item każdy pakunek ma objętość mniejszą niż dowolna ciężarówka.
	\end{itemize}
\end{frame}

\section{Model matematyczny}
\subsection{Dane wejściowe}
\subsection{Otrzymany wynik}
\section{Niewielki Przyklad}
\section{Algorytm}
\begin{frame}
\subsection{Opis algorytmu}
\frametitle{Opis algorytmu}
\end{frame}
\section{Obliczenia testowe}
\end{document}

Sprawozdanie:
	1. krótki opis problemu
	2. uwagi o algorytmie (źródło)
	3. przykład działania algorytmu
	4. źródło danycg, wyniki obliczeń i ich analiza
	5. kod źródłowy
	6. opis uruchomienia, przykładowe dane testowe
